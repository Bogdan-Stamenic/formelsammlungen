\section{Lineare Operatoren}
Im Folgenden sind $X$ und $Y$ beliebige Banachräume\defref{def:banachraum} über $\complex$.

Wir betrachten Abbildungen:
\begin{align*}
  &A: \mathcal{M} \to \mathcal{Y}\\
  &A: x \to y
\end{align*}

die Elemente $x \in \mathcal{M}$ ein Element $y \in \mathcal{Y}$ zuordnen.
\begin{itemize}
        \item $\mathcal{M}$ heißt \textbf{Domain} von $A$,
        \item $R(A) = \{y \in \mathcal{Y}: y = A(x), x \in \mathcal{M}\} \subset \mathcal{Y}$ heißt \textbf{Bildraum} (Range) von $A$.,
        \item $N(A) = \{x \in \mathcal{M} : A(x) = 0\} \subset \mathcal{M}$ heißt \textbf{Nullraum} (Kern) von A.
\end{itemize}

Wir betrachten meist Abbildungen mit $\mathcal{M} = \mathcal{X}$.

Lass $A : \X \to \Y$:
\begin{description}
        \item[Injektiv] Zu jedem $y \in R(A)$ existiert ein $x \in \X$, so dass $y = A(x)$ (eindeutige Abbildung).
        \item[Surjektiv] Wenn $R(A) = \Y$ (vollständige Abbildung).
        \item[Bijektiv] wenn injektiv und surjektiv.
\end{description}

\begin{boringDef}{Lineare Abbildung}{lineare_abbildungen}
  Eine Abbildung $A : \X \to \Y$ heißt linear, falls $A(\alpha x_{1} + \beta x_{2}) = \alpha A(x_{1}) + \beta A(x_{2}) $ für alle $x_{1}, x_{2} \in \X$ und alle $\alpha, \beta \in \complex$.
  \textbf{Bemerkung:} Lineare Abbildungen werden auch als \textit{Operatoren} bezeichnet.
\end{boringDef}

\begin{mytheo}{Lineare Abbildungen}{lineare_abbildungen}
  Sei $A : \X \to \Y$ linear, so ist $A$ genau dann injektiv, wenn $\mathcal{N}(A) = \{0\}$.
\end{mytheo}

\begin{boringDef}{Beschränktheit und Stetigkeit}{beschraenkheit_und_stetigkeit}
Seien $\X$ und $\Y$ zwei normierte Räume und $A: \X \to \Y$  eine Abbildung.
\begin{itemize}
  \item $A$ heißt \textbf{beschränkt} falls eine Konstante $C < \infty$ existiert, so dass
  \begin{equation*}
  \|A(x)\|_{\Y} \leq C, \quad \forall \|x\|_{\X} \leq 1
  \end{equation*}
  \item $A$ heißt \textbf{stetig am Punkt} $x_0 \in \X$ falls zu jedem $\epsilon > 0$ ein $\delta > 0$ existiert, so dass
  \begin{align*}
  &{\|A(x) - A(x_0)\|}_{\Y} < \epsilon,\\
  &\forall x \in \X \text{ mit } {\|x - x_0\|}_{\X} < \delta.
  \end{align*}
\end{itemize}
  $A$ heißt \textbf{stetig}, wenn $A$ an allen $x \in \X$ stetig ist.
\end{boringDef}

\paragraph{Lemma}
Seien $\X$ und $\Y$ normierte Räume und $A: \X \to \Y$.
Wenn $A$ \textit{linear} ist, so sind die folgenden Aussagen äquivalent:
\begin{itemize}
  \item $A$ ist stetig am Punkt $0$,
  \item $A$ ist stetig,
  \item $A$ ist beschränkt,
  \item Es existiert eine positive Konstante $C$, so dass
  \begin{equation*}
    {\|Ax\|}_{\Y} \leq C {\|x\|}_{\X} \quad x \in \X.
  \end{equation*}
\end{itemize}
