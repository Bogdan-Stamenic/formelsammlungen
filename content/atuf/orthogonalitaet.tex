\section{Orthogonalität}

\begin{boringDef}{Othogonalität}{orthogonalitaet}
  Sei $\Hilb$ ein Hilbertraum\defref{def:hilbertraum}.
  Zwei Vektoren $f, g \in \Hilb$ heißen orthogonal zueinander ($f \perp g$) falls ${\langle f, g \rangle}_{\Hilb} = 0$.

  Eine Folge $s = \{s_{n}\}_{n=1}^{\infty} \subset \Hilb$ heißt \textbf{Orthogonalfolge} falls $\langle s_{n}, s_{m} \rangle = 0$ für alle $n \neq m$.

  Eine Orthogonalfolge $s$ heißt \textbf{Orthonormalfolge} falls $\| s_{n}\|_{\Hilb} = 1$ für alle $n \in \natural$.
\end{boringDef}

\begin{recipe}{Innere Produkt und Orthogonalität}
Sei $s = \{s_{n}\}_{n=1}^{N}$ eine endliche Orthonormalfolge in einem Hilbertraum $\Hilb$, seien $c = \{c_{n}\}_{n=1}^{N} \subset \F$  und $d = \{d_{n}\}_{n=1}^{N} \subset \F$ zwei Zahlenfolgen, und seien
\begin{align*}
  f = \sum\limits_{n=1}^{N} c_{n} s_{n} \quad g = \sum\limits_{n=1}^{N} d_{n} s_{n},
\end{align*}

dann gilt

\begin{align}
  {\|f\|}_{\Hilb}^{2} = \sum\limits_{n=1}^{N} | c_{n}|^{2},\\
  \langle f, g \rangle = \sum\limits_{n=1}^{N} c_{n} \overline{d_{n}},
\end{align}
\end{recipe}

\paragraph{Beste Approximation in einer Orthonormalfolge}
Sei $\Hilb$ ein Hilbertraum und $\{s_{n}\}_{n=1}^{N}\subset \Hilb$ eine Orthonormalfolge.
Sei $f \in \Hilb$ beliebig.
Wir suchen eine Approximation $\tilde{f}$ von $f$ mit der Form
\begin{equation}
  \tilde{f} = \sum\limits_{n=1}^{N}c_{n}s_{n}, \quad c_{n} \in \F
\end{equation}
so dass
\begin{equation*}
  \| f - \tilde{f} \|_{\Hilb}
\end{equation*}
so klein wie möglich wird.

\begin{equation*}
  \|f - \tilde{f}\|^{2} = \langle f - \tilde{f}, f - \tilde{f} \rangle
\end{equation*}
\begin{align*}
  &= \|f\|_{\Hilb}^{2} - \sum\limits_{n=1}^{N}\overline{c_{n}} \langle f, s_{n} \rangle - \sum\limits_{n=1}^{N}c_{n} \overline{\langle f, s_{n}\rangle}  + {\left\|\sum\limits_{n=1}^{N}\langle c_{n}, s_{n} \rangle \right\|}^{2}\\
  &= \|f\|_{\Hilb}^{2} + \sum\limits_{n=1}^{N}\left|\langle f, s_{n} \rangle - c_{n}\right|^{2} - \sum\limits_{n=1}^{N}|\langle f, s_{n} \rangle |^{2}
\end{align*}

$f - \tilde{f}$ minimieren:
\begin{tcolorbox}[title={Kooeffizienten für die Approximation}, colback=white!, colframe=black!40]
\begin{equation}
  c_{n} = \langle f, s_{n} \rangle_{\Hilb}, \quad f \in \Hilb
\end{equation}
mit einer quadratischen Fehler von:
\begin{equation}
  \|f - \tilde{f}\|^{2} = \|f\|^{2} - \sum\limits_{n=1}^{N}|\langle f,s_{n} \rangle |^{2}.
\end{equation}
\end{tcolorbox}

\begin{mytheo}{Bessel Gleichungen}{bessel}
  Sei $\{s_{n}\}_{n=1}^{N}$ eine Orthonormalfolge in einem Hilbertraum $\Hilb$, so gilt für alle $f \in \Hilb$
  \begin{enumerate}
    \item \textbf{Bessel'sche Ungleichung}
          \begin{equation}
            \sum\limits_{n=1}^{N} |\langle f, s_{n} \rangle|^{2} \leq \| f \|^{2}\label{eq:besselsche_ungleichung}
          \end{equation}
          \item \textbf{Bessel'sche Gleichung}
          \begin{align}
            &\left\| f - \sum\limits_{n=1}^{N} \langle f, s_{n} \rangle  s_{n}\right\|^{2}\nonumber\\
            = &\|f\|^{2} - \sum\limits_{n=1}^{N} |\langle f, s_{n} \rangle|^{2} \geq 0\label{eq:besselsche_gleichung}
          \end{align}
  \end{enumerate}
\end{mytheo}

Sei $\{s_{n}\}_{n=1}^{N}$ eine Orthonormalfolge\defref{def:orthogonalitaet} in einem Hilbertraum $\Hilb$, und sei
\begin{equation}
S_{n} = \mathrm{span}\{s_{n}\}_{n=1}^{N} = \left\{f = \sum\limits_{n=1}^{N} c_{n}s_{n} : c_{n} \in \F\right\} \subset \Hilb \label{eq:hilbert_subspace}
\end{equation}

der lineare Unterraum\defref{def:unterraum} aller Linearkombinationen von $\{s_{n}\}_{n=1}^{N}$.

Wir hatten gesehen, dass $\tilde{f} = \sum\limits_{n=1}^{N} \langle f, s_{n} \rangle s_{n}$ erfüllt
\begin{equation}
  \tilde{f} = \mathrm{arg}\min_{\substack{g\in S_{n}}} \| f - g \|. \label{eq:approximation}
\end{equation}

Sei $v = f - \tilde{f}$ der minimale Fehler, so gilt
\begin{align*}
  \langle v, s_{n} \rangle &= \langle f - \tilde{f}, s_{n} \rangle =\langle f, s_{n} \rangle - \langle \tilde{f}, s_{n} \rangle\\
  &= \langle f, s_{n} \rangle - \left\langle \sum\limits_{k=1}^{N}\langle f, s_{k} \rangle , s_{n} \right\rangle\\
  &= \langle f, s_{n} \rangle -  \sum\limits_{k=1}^{N}\langle f, s_{k} \rangle  \langle s_{k}, s_{n} \rangle\\
  &= \langle f, s_{n} \rangle - \langle f, s_{n} \rangle\\
  &= 0, \quad \forall n=1,2,\dots,N
\end{align*}

\paragraph{Orthogonalitätsprinzip}
Sei $\Hilb$ ein Hilbertraum\defref{def:hilbertraum} und $s = \{s_{n}\}_{n=1}^{N}$ eine Orthonormalfolge, und sei $S_{n} \subset \Hilb$ durch \eqref{eq:hilbert_subspace} gegeben.
Zu jedem beliegen $f \in \Hilb$ sei $\tilde{f}$ durch~\eqref{eq:approximation} gegeben, dann gilt
\begin{equation}
  \langle f - \tilde{f}, s \rangle  = 0, \quad \forall s \in S_{n}
\end{equation}

\begin{boringDef}{Dimension eines Hilbertraumes}{dimension_eines_hilbertraumes}
  Sei $\Hilb$ ein Hilbertraum\defref{def:hilbertraum}. Wir sagen, $\Hilb$ ist $N$-dimensional, falls ein $N \in \natural$ und ein Orthogonalsystem ${\{s_{n}\}}_{n=1}^{N} \subset \Hilb$ existiert, so dass für alle $f \in \Hilb$ gilt
  \begin{align}
    &\left\| f - \sum\limits_{n=1}^{N} \langle f, s_{n} \rangle s_{n} \right\|_{\Hilb} = 0\\
    \implies &f = \sum\limits_{n=1}^{N} \langle f, s_{n} \rangle s_{n}
  \end{align}
  In diesem Fall heißt ${\{s_{n}\}}_{n=1}^{N}$ \textbf{Orthonormalbasis (ONB)} für $\Hilb$.
  Andernfalls, falls keine Orthonormalbasis mit Dimension $N<\infty$ existiert, sagen wir, dass $\Hilb$ $\infty$-dimensional ist.
\end{boringDef}

\begin{recipe}{Nützliche Lemma}
Sei $s = \{s_{n}\}_{k=1}^{N}$ eine Orthonormalfolge in einem Hilbertraum $\Hilb$. Zu jedem $f \in \Hilb$ existiert ein $\tilde{f}$ so dass
\begin{align}
  \lim_{\substack{N \to \infty}} &\left\| \tilde{f} - \sum\limits_{n=1}^{N} \langle f, s_{n} \rangle s_{n} \right\| = 0\\
  &\implies \sum\limits_{n=1}^{N} \langle f, s_{n} \rangle s_{n} = \tilde{f}.
\end{align}

\textbf{Bemerkung:} Im Allgemeinen ist $\tilde{f} \neq f$.
\end{recipe}

\begin{boringDef}{Orthonormalbasis}{orthonormalbasis}
  Eine Orthonormalfolge $s = {\{s_{n}\}}_{n=1}^{\infty}$ in einem Hilbertraum $\Hilb$ heißt \textbf{vollständige Orthonormalfolge} oder \textbf{Orthonormalbasis} falls
  \begin{equation}
    \lim_{\substack{N \to \infty}} {\left\| f - \sum\limits_{n=1}^{N} \langle f, s_{n} \rangle s_{n} \right\|}_{\Hilb}= 0, \quad \forall f \in \Hilb.
  \end{equation}
\end{boringDef}


\begin{mytheo}{Parseval'sche Gleichung}{parseval}
  Eine Orthogonalfolge $s = {\{s_{n}\}}_{n=1}^{\infty}$ in einem Hilbertraum $\Hilb$ ist genau dann vollständig, wenn
  \begin{equation}
    \| f \|_{\Hilb}^{2} = \sum\limits_{n=1}^{\infty} | f, s_{n} |_{\Hilb}^{2}, \quad \forall f \in \Hilb.
  \end{equation}
  \textbf{Bemerkung:} Nicht jeder ($\infty$-dimensionale) Hilbertraum besitzt eine Orthonormalbasis.
\end{mytheo}

\begin{boringDef}{Ich brauche einen Namen}{balh}
  Ein Hilbertraum $\Hilb$ heißt \textbf{seperabel}, wenn es ein vollständiges Orthogonalsystem (eine ONB) besitzt.
\end{boringDef}
