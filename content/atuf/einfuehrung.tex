		\section{Einführung}
        \begin{itemize}
                \item Frames sind eine Verallgemeinerung von Orthogonalbasen $\{s_{n}\}_{n \in \mathbb{Z}} \subset H$
          \item Signale können in diese Frames entwickelt werden:
                \begin{equation*}
                  f(t) = \sum\limits_{n=-\infty}^{\infty} \langle f, s_{n} \rangle s_{n}
                \end{equation*}
                \begin{definition}{Funktionen endlicher Energie $L^{2}(\mathbb{R})$}
                $L^{2}(\mathbb{R})$ definiert die Menge aller Funktionen endlicher Energie, d.h.\:
                \begin{equation*}
                  {\|f(t)\|_{2}}^{2} = \int\limits_{-\infty}^{\infty} |f(t)|^{2}\,\mathrm{d}t < \infty
                \end{equation*}
                \end{definition}
          \item Fourier Transformation für $f \in L^{2}(\mathbb{R})$:
                \begin{align*}
                  \hat{f}(\omega) &= \int\limits_{-\infty}^{\infty} f(t) \, e^{-i\omega t}\,\mathrm{d}t\\
                  f(t) &= \dfrac{1}{2\pi} \int\limits_{-\infty}^{\infty} \hat{f}(\omega) \, e^{i\omega t}\,\mathrm{d}\omega
                \end{align*}
          \item Parseval'sche Formel:
                \begin{equation*}
                  {\|\hat{f}\|_{2}}^{2} = 2\pi {\|f\|_{2}}^{2}
                \end{equation*}
                \begin{definition}{Bandbegrenzte Funktionen}
                  \begin{align*}
                    \mathrm{PW}(\Omega) = \{&f \in L^{2}(\mathbb{R}): \hat{f}(\omega) = 0,\\
                    &\forall|\omega| > \dfrac{\Omega}{2}\},
                  \end{align*}
                  wobei $\Omega$ die Bandbreite ist.
                \end{definition}
                \begin{definition}{Nyquist-Shannon Sampling Theorem}
                  \begin{align*}
                    f(t) &= \sum\limits_{-\infty}^{\infty} f(k\cdot T) \cdot \mathrm{sinc}\left(\dfrac{\Omega}{2} [t - k\,T]\right),\\
                    &= \sum\limits_{k} f(k\cdot T) \cdot s_{k}(t),\\
                  \end{align*}
                  wobei $\Omega$ die Bandbreite, $T$ das Periodendauer ist:
                  \begin{equation*}
                    T = \dfrac{2\pi}{\Omega}.
                  \end{equation*}
                  Solange $\mathrm{sinc}(t - k\,T)$ eine Orthonormalbasis bildet \textit{und} $f(t)$ zuverlässig zum Zeitpunkt $k\cdot T$ abgetastet wird, kann $f(t)$ vollständig rekonstruieren.
                \end{definition}
                \item \textbf{Problem:} Wenn ein Signal nicht ideal abgetastet wird bzw. die $\{s_{k}(t)\}$ keine ONB bildet, dann gilt das Nyquist-Shannon Theorem nicht mehr!
        \end{itemize}
