\section{Adjungierte Operator}

\begin{mytheo}{Adjungierte Operator}{adj_operator}
  Seien $\mathcal{K}$ und $\Hilb$ Hilberträume und $A \in \mathcal{B}(\mathcal{K}, \Hilb)$ beliebig, so existiert ein eindeutiger Operator $A^{*} \in \mathcal{B}(\Hilb, \mathcal{K})$ so dass
  \begin{equation}
    {\langle Af, g \rangle}_{\Hilb} = {\langle f, A^{*}g \rangle}_{\mathcal{K}}, \quad \forall f \in \mathcal{K}, g \in \Hilb\label{eq:adj_operator}
  \end{equation}
  und $\| A^{*} \| = \| A \|$.
  $A^{*}$ heißt der zu $A$ adjungierte Operator.
\end{mytheo}

\begin{boringDef}{Postiver Operator}{pos_operator}
  Sei $A \in \B(\Hilb)$ selbstadjungiert, so heitßt $A$ \textbf{positiv} falls
  \begin{equation}
    \langle Af, f \rangle \geq 0, \quad \forall f \in \Hilb.\label{eq:pos_operator}
  \end{equation}
  In dem Fall, schreibt man auch $A \geq 0$.
\end{boringDef}
