\section{Besselfolgen}

Eine Frame ${\{s_n\}}_{n=1}^\infty$ ist eine verallgemeinerte (Quasi- )Basis.

\begin{boringDef}{Besselfolge}{besselfolge}
  Eine Folge $\{s_n\}_{n=1}^\infty \subset \Hilb$ in einem Hilbertraum $\Hilb$ heißt \textbf{Besselfolge}, falls zu eine Konstante $B > 0$ existiert, so dass
  \begin{equation}
    \sum\limits_{n=1}^\infty \left|\langle f, s_n \rangle\right|^2 \leq B {\|f\|}_{\Hilb}^{2} \quad \forall f \in \Hilb.\label{eq:besselfolge}
  \end{equation}
  Die Konstante $B$ heißt \textbf{Besselschranke} von ${\{s_n\}}_{n=1}^\infty$, und die kleinste Konstante $B$, für die \eqref{eq:besselfolge} gilt ist die \textbf{optimale} Besselschranke von ${\{s_n\}}_{n=1}^\infty$.

  \textbf{Bemerkung:} Jede ONB ist Besselfolge mit opt.\ Besselschranke 1.
\end{boringDef}

\begin{mytheo}{Beschränkter Operator auf Besselfolgen}{beschraenkter_operator_aus_besselfolgen}
  Sei $s = {\{s_n\}}_{n=1}^\infty$ eine Besselfolge in $\Hilb$ mit Besselschranke $B_s$, so ist
  \begin{equation*}
    T_s : {\{c_n\}}_{n=1}^\infty \mapsto \sum\limits_{n=1}^\infty c_n s_n
  \end{equation*}
  ein beschränkter Operator $T_s : l^2 \to \Hilb$ mit Norm $\|T_s\| \leq \sqrt{B_s}$.
\end{mytheo}

\begin{boringDef}{Operatoren auf Besselfolgen}{operatoren_auf_besselfolgen}
  Sei $s = {\{s_n\}}_{n=1}^\infty$ eine Besselfolge\defref{def:besselfolge} in $\Hilb$, so assozieren wir mit $s$ die folgenden Operatoren:
  \begin{enumerate}
    \item \textbf{Analysisoperator} $S_s : \Hilb \to l^2$
    \item \textbf{Syntheseoperator}
    \item \textbf{Frameoperator}
    \item \textbf{Gramoperator}
  \end{enumerate}
\end{boringDef}
