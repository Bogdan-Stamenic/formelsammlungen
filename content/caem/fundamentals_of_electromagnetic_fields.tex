
\section{Fundamentals of Electromagnetic Fields}
\subsection{Maxwell's Equations}
Maxwell's Equations (integral form):
\begin{align*}
  \oint\limits_{C(A)} \vec{h}(\vec{r}, t)\cdot\mathrm{d}\vec{s}
  &= \iint\limits_{A}\left[ \vec{j}(\vec{r}, t) + \dfrac{\mathrm{d}\vec{d}(\vec{r}, t)}{\mathrm{d}t} \right]\cdot\mathrm{d}\vec{a}\\
  \oint\limits_{C(A)} \vec{e}(\vec{r}, t)\cdot\mathrm{d}\vec{s}
  &= -\iint\limits_{A}\left[ \vec{m}(\vec{r}, t) + \dfrac{\mathrm{d}\vec{b}(\vec{r}, t)}{\mathrm{d}t} \right]\cdot\mathrm{d}\vec{a}\\
  \oiint\limits_{A(V)}\vec{d}(\vec{r}, t)\cdot\mathrm{d}\vec{A} &= \iiint\limits_{V}\rho(\vec{r},t)\,\mathrm{d}v\\
  \oiint\limits_{A(V)}\vec{b}(\vec{r}, t)\cdot\mathrm{d}\vec{A} &= \iiint\limits_{V}\rho_{m}(\vec{r},t)\,\mathrm{d}v
\end{align*}

Maxwell's Equations (differential form):
\begin{align*}
&\nabla\times\vec{h}(\vec{r}, t) = \vec{j}(\vec{r},t) + \dfrac{\mathrm{d}\vec{d}(\vec{r},t)}{\mathrm{d}t}\\
&\nabla\times\vec{e}(\vec{r}, t) = -\dfrac{\mathrm{d}\vec{b}(\vec{r},t)}{\mathrm{d}t} -\vec{m}(\vec{r},t)\\
&\nabla\cdot\vec{d}(\vec{r},t) = \rho(\vec{r},t)\\
&\nabla\cdot\vec{b}(\vec{r},t) = \rho_{m}(\vec{r},t)
\end{align*}

Material Relations:
\begin{align*}
&\vec{d}(\vec{r},t) = \epsilon(\vec{r},t) * \vec{e}(\vec{r},t)\\
&\vec{j}(\vec{r},t) = \sigma(\vec{r},t) * \vec{e}(\vec{r},t) + \vec{j}_{\mathrm{exc}}(\vec{r},t)\\
&\vec{b}(\vec{r},t) = \mu(\vec{r},t) * \vec{h}(\vec{r},t)\\
&\vec{m}(\vec{r},t) = \vec{m}_{\mathrm{exc}}(\vec{r},t)\\
\end{align*}

Poisson Equation ($\partial_{t}\vec{e} = 0$, $\epsilon(\vec{r}) = \mathrm{const.}$):
\begin{equation*}
\nabla\cdot\left(\nabla\phi\right) = \Delta\phi = -\dfrac{\rho(\vec{r})}{\epsilon}
\end{equation*}

\subsection{Green's Theorems and Uniqueness of the Poisson Equation}
\todo{Add Green's theorems\\}
\begin{info}{Uniqueness of Poisson Equation}
  Solutions of $\Delta\phi(\vec{r}) = -\rho(\vec{r})/\epsilon$ are uniquely defined by specifying:
  \begin{enumerate}
    \item $\rho(\vec{r})$ and $\left.\phi(\vec{r})\right|_{\vec{r}\in F(V)}$
    \item \textbf{or} $\rho(\vec{r})$ and $\left.\dfrac{\partial}{\partial n}\phi(\vec{r})\right|_{\vec{r}\in F(V)}$
    \item \textbf{or} $\rho(\vec{r})$ and $\left.\phi(\vec{r})\right|_{\vec{r}\in F_{1}(V)}$ and $\left.\dfrac{\partial}{\partial n}\phi(\vec{r})\right|_{\vec{r}\in F_{2}(V)}$\\
          with $F_{1}(V) \cup F_{2}(V) = F(V)$,\\
          $F_{1}(V) \cap F_{2}(V) = \emptyset$
  \end{enumerate}
\end{info}

\subsection{Uniqueness of Time-Harmonic Electromagnetic Fields}
Unique solutions while assuming arbitrarily small losses, when:
\begin{enumerate}
\item Presetting $\vec{n}\times\vec{E}$ on boundary surface $A$
\item Presetting $\vec{n}\times\vec{H}$ on $A$
\item Presetting $\vec{n}\times\vec{E}$ on parts of $A$, presetting $\vec{n}\times\vec{H}$ on the remainder of $A$
\item In general, it is also possible to define $\vec{n}\times\vec{E}$ as a function of $\vec{n}\times\vec{H}$ or vice-versa
\end{enumerate}

\subsection{The Dirac Delta-Functional}
\begin{itemize}
  \item Motivated by volume charge density representation of a point charge:
      \begin{equation*}
        \iiint\limits_{V}\rho(\vec{r}, \vec{r_{P}})\mathrm{d}v =
        \begin{cases}
          Q_{P} \quad \text{for }\vec{r_{P}}\in V\\
          0 \quad \text{elsewhere}
        \end{cases}
      \end{equation*}
  \item Dirac delta defined via functional equation
        \begin{equation*}
\int\limits^{+\infty}_{-\infty}\delta(x)\varphi(x)\,\mathrm{d}x = \varphi(x)
        \end{equation*}
  \item A functional maps \textbf{a function onto a number}
        \begin{equation*}
\varphi(x) \mapsto \varphi(0)
        \end{equation*}
  \item A functional is linear, when
        \begin{align*}
&F\{\varphi_{1}+\varphi_{2}\} = F\{\varphi_{1}\} + F\{\varphi_{2}\}\\
&F\{k\varphi\} = kF\{\varphi\}
        \end{align*}
        \item Domain of functionals are \textit{function spaces}
        \item Function space is in general a set of functions and maps from a set $X$ onto a set $Y$
        \item In many cases, vector addition and scalar multiplication can be defined. Such function spaces are called \textit{vector spaces}.
        \item Properties of $\delta_{n}(x)$:
        \begin{align*}
          &\delta_{n}(ax) = \dfrac{1}{|a|} \delta(x)\\
          &\int\limits^{+\infty}_{-\infty}\delta^{(n)}\varphi(x)\,\mathrm{d}x = (-1)^{n}\varphi^{(n)}(0)\\
          &H^{\prime}(x) = \delta(x),\quad H(x) =
          \begin{cases}
            1 \quad x\geq0\\
            0 \quad \text{elsewhere}
          \end{cases}
        \end{align*}
\end{itemize}
\subsubsection{Multi-dimension $\delta$-distributions in different coordinate systems}
\begin{enumerate}
\item Cartesian coordinates $P=(x',y',z')$
        \begin{equation*}
          \mathrm{d}v = \mathrm{d}x\,\mathrm{d}y\,\mathrm{d}z
        \end{equation*}
        \begin{align*}
          \delta(\vec{r}-\vec{r}_{P}) = \delta(x-x')\delta(y-y')\delta(z-z')
        \end{align*}
\item Cylinder coordinates $P=(\rho',\varphi',z')$
        \begin{equation*}
          \mathrm{d}v = \rho\,\mathrm{d}\rho\,\mathrm{d}\varphi\,\mathrm{d}z
        \end{equation*}
        \begin{align*}
          &\delta(\vec{r}-\vec{r}_{P}) =\\
          &\begin{cases}
            \dfrac{1}{\rho}\delta(\rho-\rho')\delta(\varphi-\varphi')\delta(z-z'), \quad \rho'\neq0\\
            \dfrac{1}{\pi\rho}\delta(\rho)\delta(z-z'), \quad \rho'=0
          \end{cases}
        \end{align*}
\item Cylinder coordinates $P=(r',\vartheta',\varphi')$
        \begin{equation*}
          \mathrm{d}v = r^{2}\,\mathrm{d}r\,\sin(\vartheta)\,\mathrm{d}\,\vartheta\mathrm{d}\varphi
        \end{equation*}
        \begin{align*}
          &\delta(\vec{r}-\vec{r}_{P}) =\\
          &\begin{cases}
            \dfrac{1}{r^{2}\sin\vartheta}\delta(r-r')\delta(\vartheta-\vartheta')\delta(\varphi-\varphi'),r'\neq0\\
            \dfrac{1}{2\pi r^{2}}\delta(r), \quad r'=0
          \end{cases}
        \end{align*}
\end{enumerate}

\subsection{Solution of Poisson Equation by Green's Function}
\begin{itemize}
  \item Green's function as solution for a Dirac delta excitation according to
        \begin{equation*}
          \Delta G(\vec{r},\vec{r}') = -\delta(\vec{r} - \vec{r}').
        \end{equation*}
  \item Green's function of \textit{two-dimensional} free space
        \begin{equation*}
          G(\vec{r},\vec{r}') = \dfrac{1}{2\pi}\ln\dfrac{1}{\sqrt{(x-x')^{2} + (y-y')^{2}}}
        \end{equation*}
\end{itemize}

\subsection{Generalization of the Solution Strategy for Non-Self-Adjoint Linear Differential Operators}
\begin{definition}{Adjoint Operator}
    \begin{equation*}
    \langle\psi(x)\mathcal{L}\,\varphi(x)\rangle = \langle\varphi(x)\mathcal{L}^{*}\psi(x)\rangle
    \end{equation*}
\end{definition}
\begin{itemize}
  \item Self-adjoint diff. operator of 2. order:
        \begin{equation*}
          \mathcal{L} = \dif{}{}{x}\left(p(x)\dif{}{}{x}\right) + g(x)
        \end{equation*}
  \item Formally adjoint lin. operator $\mathcal{L}^{\dagger}$ to $\mathcal{L}$:
        \begin{align*}
          &\mathcal{L} = \sum\limits^{n}_{\nu=0} a_{\nu}(x) \dif{\nu}{}{x},\\
          &\mathcal{L}^{\dagger} = \sum\limits^{n}_{\kappa=0} (-1)^{\kappa} \sum\limits^{\kappa}_{\nu=0}
          \begin{pmatrix}\kappa\\ \nu\end{pmatrix}
          \left(\dif{(\kappa-\nu)}{a_{\kappa}(x)}{x}\right)
          \dif{\nu}{}{x}
        \end{align*}
\end{itemize}

\subsection{Dyadic Green's Functions}
\begin{itemize}
        \item A Green's function is the field of a Delta function source distribution (e.g. field of Hertzian Dipole).
        \item Characterizes a particular solution domain, similar to how a LTI-system is characterized by its impulse response.
  \item Field of arbitrary source distribution as integral over source distribution:
        \begin{align*}
          \vec{E}(\vec{r}) = \iiint\dyade{G}{\vec{r},\vec{r}'} \cdot \vec{J}(\vec{r}')\,\mathrm{d}v'
        \end{align*}
\end{itemize}

\subsection{Radiation from Electromagnetic Sources}
\begin{itemize}
        \item Assume homogenous, isotropic (free) space and \textit{electric current} sources:
        \begin{align*}
          &\vec{B} = \nabla\times\vec{A}\\
          &\vec{A} = \nabla\times\vec{C} + \nabla\xi \text{(Helmholtz theorem)}\\
          \implies &\vec{E} = -j\omega\vec{A} -\nabla\phi
        \end{align*}
  \item Using the \textit{Lorenz-Gauge} $\nabla\cdot\vec{A} + j\omega\mu\epsilon\phi = 0$, we get the \textbf{Helmholtz-Equations}:
        \begin{align*}
          \Delta\vec{A} + k^{2}\vec{A} &= -\mu\vec{J}\text{ (vector)}\\
          \Delta\phi + k^{2}\phi &= \dfrac{\nabla\cdot\vec{J}}{j\omega\epsilon} = -\dfrac{\rho}{\epsilon}\text{ (scalar)}
        \end{align*}
\end{itemize}
\begin{info}{Green's functions in Homogenous Space}
  Integrating over the source distribution $\hat{J}$ in homogenous space returns the field quantity $\hat{E}$:
  \begin{equation*}
    \hat{E}(\vec{r},\vec{r}') = \iiint\limits_{V}\dyade{G}_{\hat{J}}^{\hat{E}}(\vec{r},\vec{r}') \hat{J}(\vec{r}')\,\mathrm{d}v'
  \end{equation*}
  when $\dyade{G}_{\hat{J}}^{\hat{E}}$ is the corresponding Green's function. For the elec. and magn. vector potentials, $\dyade{G}$ would be:
  \begin{align*}
    \dyade{G}_{J}^{A} = \dfrac{\mu}{4\pi} \dfrac{e^{-jk|\vec{r}-\vec{r}'|}}{|\vec{r}-\vec{r}'|},\\
    \dyade{G}_{M}^{F} = \dfrac{\epsilon}{4\pi} \dfrac{e^{-jk|\vec{r}-\vec{r}'|}}{|\vec{r}-\vec{r}'|}.
  \end{align*}
  For the elec. and magn. fields, $\dyade{G}$ would be:
  \begin{align*}
    \dyade{G}_{J}^{E} = \dfrac{-j}{4\pi}\left(\dyade{I} + \dfrac{1}{k^{2}}\nabla\nabla\right)\dfrac{e^{-jk|\vec{r}-\vec{r}'|}}{|\vec{r}-\vec{r}'|},\\
    \dyade{G}_{M}^{H} = \dfrac{-j}{4\pi}\left(\dyade{I} + \dfrac{1}{k^{2}}\nabla\nabla\right)\dfrac{e^{-jk|\vec{r}-\vec{r}'|}}{|\vec{r}-\vec{r}'|}.
  \end{align*}
\end{info}

\subsection{Huygen's Principle}
\begin{recipe}{Huygen's Principle}
  \begin{enumerate}
    \item Define a source volume $V_{a}$ and a solution volume $V_{b}$.
    \item Enclose all elec./magn. current sources in $V_{a}$.
    \item Calculate
          \begin{align*}
            &\vec{J}_{a}(\vec{r}') = \vec{n} \times \vec{H}(\vec{r}'),\\
            &\vec{M}_{a}(\vec{r}') = -\vec{n} \times \vec{E}(\vec{r}'),
          \end{align*}
          for the surface normal $\vec{n}$ of the volume $V_{a}$.
    \item All field in $V_{a}$ vanish when $\vec{J}_{a}$ and $\vec{M}_{a}$ are applied to $A(V_{a})$.
    \item We can \textit{formally} replace the material in $V_{a}$ without affecting the fields in $V_{b}$. $\implies$ Choose free space.
  \end{enumerate}
\end{recipe}
